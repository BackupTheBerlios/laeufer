\chapter{Serielle Protokollschicht: LSP-Protokoll, Version 0}\label{slp-proto}

Alle folgenden Aussagen beziehen sich auf Version 0 des Protokolls, mit Ausnahme
explizit gekennzeichneter Stellen.\\

Das LSP-Protokoll transferiert ``namenlose'' Datenbl\"ocke unspezifizierter
L\"ange \"uber eine serielle Verbindung. Es spezifiziert ``Blocks'', die die
atomaren Einheiten der \"Ubertragung darstellen (und deren Semantik dar\"uberliegenden
Protokollschichten \"uberl\"a\ss t) und stellt sicher, da\ss\ f\"ur eine fehlerfreie
Leitung die eintreffenden Bl\"ocke eindeutig erkannt werden k\"onnen.

Zur \"Ubertragung wird auf 8-Bit-Zeichen zur\"uckgegriffen.

\section{Zeichenvorrat}
\begin{tabular}{l|c|l}
{\bf Code} & {\bf Name} & {\bf Bedeutung} \\
\hline
{\tt 0x00--0xef} & & Freie Zeichen \\
{\tt 0xf0} & {\tt SOB} & Blockstart \\
{\tt 0xf1} & {\tt EOB} & Blockende \\
{\tt 0xf2} & {\tt ESC} & Escape \\
{\tt 0xf3--0xf7} & -- & {\it reserviert} \\
{\tt 0xf8--0xff} & & Freie Zeichen \\
\end{tabular}

Der Zeichenvorrat teilt sich in die Menge der Freien Zeichen, und in
die Menge der Kontrollzeichen $\{${\tt 0xf0 \ldots 0xf7}$\}$.

\section{Block}
Ein {\it Block} ist eine Sequenz von Zeichen aus dem Zeichenvorrat, deren erstes
Zeichen ein {\tt SOB} und deren letztes Zeichen ein {\tt EOB} ist, und f\"ur die
gilt, da\ss\ sie ansonsten keine weiteren Vorkommnisse von Kontrollzeichen mit Ausnahme
von {\tt ESC} wie in \ref{j-lsp-encoding} angegeben.

Blocks werden in der Reihenfolge, in der sie definiert sind, sequentiell uebertragen und
empfangen.

\section{Nachrichtenkodierung}\label{j-lsp-encoding}
Um eine beliebige Folge auf dem kompletten Zeichenvorrat in einen Block
zu kodieren, wird vorgegangen wie folgt:

\subsection{Blockbeginn}
Zu Beginn der  Folge wird ein {\tt SOB} angegeben.

\subsection{Elemente des Zeichenvorrates}
F\"ur alle Zeichen aus der Sequenz wird in der gegebenen Reihenfolge folgendes
in den Block eingetragen:
\begin{itemize}
  \item{{\tt b}, falls das zu \"ubertragende Zeichen {\tt b} und ein Element der
    Freien Zeichen ist.}
  \item{{\tt ESC $\text{\tt b}^\prime$}, falls das \"ubertragende Zeichen {\tt b} kein
    Element der Freien Zeichen ist und $\text{\tt b}^\prime = b \mod \text{\tt 0x10}$.
  }
\end{itemize}

\subsection{Blockende}
Nachdem alle Elemente der Folge eingetragen wurden, wird {\tt EOB} eingetragen.


\section{Nachrichtendekodierung}
Um einen Block in die urspruengliche Folge zu dekodieren, wird invers zu \ref{j-lsp-encoding}
vorgegangen. Bei allen Zeichen, die einem {\tt ESC} folgen, werden dabei die h\"ochstwertigen
vier Bit gesetzt, um das urspr\"ungliche Zeichen wiederherzustellen.

