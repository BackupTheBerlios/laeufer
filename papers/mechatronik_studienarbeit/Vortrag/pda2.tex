
\begin{slide}{R�ckweg der Nachricht}
  \begin{itemize}
  \item Die Nachricht kommt als \texttt{Message} beim
    \texttt{message\_dispatcher} an.
  \item �berpr�fung der Nachricht (ID, Befehl)
  \item �bergabe an Treiber �ber \texttt{receive()}
  \item Reaktion des Treibers.
  \end{itemize}
\end{slide}


% ============================================================


\overlays{3}{%
  \begin{slide}{Kommunikation mit der GUI}
    \untilSlide*{1}{
      Verwendung von Signals und Slots aus QT
      \begin{itemize}
      \item Treiber definiert seine Signale.
      \item GUI verbindet Signale mit Slots von GUI-Elementen,
        Methoden, anderen Treibern etc.
      \item $\to$ Fahrzeuglogik wird durch Routing der Signale
        implementiert.
      \end{itemize}
    }
    \fromSlide*{2}{
      Hier: 
      \begin{itemize}
      \item Treiber wirft Signal \texttt{birne\_defekt}
      \item GUI verbindet dies mit Warnelement auf Display
      \end{itemize}
      \onlySlide{3} {Drawback: Festlegung auf QT und Pr�prozessor.}
    }
  \end{slide}
  
}