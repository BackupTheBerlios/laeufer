\overlays{3}{%
\begin{slide}{Personen}
  \begin{itemstep}
    \item Christoph Reichenbach, Informatik, 9. Semester
    \item Christian R�diger, Informatik, 7. Semester
    \item Markus Weimer, Wirtschaftsinformatik, 7. Semester
  \end{itemstep}
\end{slide}
}

\overlays{4}{%
\begin{slide}{Umfeld: Projekt L�ufer}
\begin{itemstep}
  \item Studentisches Projekt
  \item Entwicklung eines ,,muskelgetriebenen Reisefahrzeugs''
  \item Industriepartnerschaften
  \item Innovative Technologien
\end{itemstep}
\end{slide}
}

\overlays{4}{%
\begin{slide}{Aufgabe dieser Arbeit}
\begin{itemstep}
  \item Entwicklung eines Mechatronik-Frameworks
  \item Platinen samt Mini-Betriebssystem
  \item Kommunikation
  \item Fahrer-Schnittstelle
\end{itemstep}
\end{slide}
}

\overlays{4}{%
\begin{slide}{Ziele}
Das erstelle Mechatronik-Framework soll
\begin{itemstep}
  \item sp�ter von den Ingenieuren handhabbar,
  \item durch Industriepartner f�rderbar,
  \item flexibel erweiterbar und
  \item vom Endanwender (L�ufer-Fahrer) akzeptierbar.
\end{itemstep}
sein.
\end{slide}
}

\overlays{4}{%
\begin{slide}{�bersicht �ber das Folgende}
Weg einer Nachricht:
\begin{itemstep}
  \item Benutzer dr�ckt Blinker-Knopf
  \item Weg durch den PDA und die Kommunikation
  \item Umsetzung des Benutzerwunsches auf der Platine
  \item Feststellung eines Fehlers samt Benachrichtigung des Fahrers
\end{itemstep}
\end{slide}
}
