%\section{Das Projekt L�ufer}

Bei diesem Fahrzeug handelt es sich um ein von seinen Entwicklern
,,L�ufer'' getauftes Tandem, das sich durch die Vielzahl an
technischen Neuerungen auszeichnet, deren Entwicklung sich zwei Zielen
unterordnet: Reisetauglichkeit und Fahrspa�.

% Projekt
Entwickelt wird das Fahrzeug von Studenten der TU Darmstadt, TU
M�nchen sowie der FH Darmstadt in einem sehr interdisziplin�ren Team.
Beteiligt sind neben Maschinenbauern verschiedener Spezialisierungen
auch (Wirtschafts-)Informatiker, Textil- und Industrie-Designer.  Ziel
des Projektes ist die Entwicklung eines muskelgetriebenen
\emph{Reise}fahrzeugs f�r zwei Personen.  Der Schwerpunkt liegt hier
auf der Reisetauglichkeit, und nicht auf Dingen wie der theoretisch
erreichbaren Maximalgeschwindigkeit.  Neben der technischen
Herausforderung sind das Lernen in anderen Strukturen sowie die
intensive Zusammenarbeit mit Partnern aus Industrie und Forschung
Triebfedern f�r das Projekt.

% Technik 
Ein wichtiger Schritt, die Ziele Fahrspa� und Reisetauglichkeit zu
erreichen, ist die Gewichtsreduktion. So ist der L�ufer komplett aus
Aluminium und Kohlefaser gefertigt, was sein Gesamtgewicht auf ca.
$50kg$ reduziert.  Der L�ufer verf�gt im Gegensatz zu vielen anderen
muskelgetriebenen Fahrzeugen �ber eine Verkleidung. Diese dient zum
einen der Verbesserung der Aerodynamik, zum anderen aber auch als
Regenschutz.  Dies zusammen mit dem Antriebsstrang, auf den weiter
unten eingegangen werden soll, erm�glicht es zwei normal trainierten
Menschen mit dem Fahrzeug eine Durchschnittsgeschwindigkeit von ca.
$40km/h$ zu erreichen.  Dieser errechnete Wert bezieht sich allerdings
nicht etwa auf eine besonders ebene Route, sondern auf eine Strecke im
Odenwald rund um Darmstadt, also im Mittelgebirge.


% �berleitung zu Christian
Insbesondere im Bereich der Mechatronik, also der Verbindung von
Mechanik mit moderner Informationstechnologie, geht der L�ufer nicht
nur f�r Fahrrad-Verh�ltnisse neue Wege.



%%% Local Variables:
%%% mode: latex 
%%% TeX-master: "master"
%%% End: