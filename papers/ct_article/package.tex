\section*{Fazit}
Im Rahmen des Projekts L�ufer ist ein Mechatronik-Framework
entstanden, dass explizit von Informatikern f�r Ingenieure entwickelt
wurde.  Dies erm�glicht es den Ingenieuren im Projekt, ohne grossen
Aufwand die Software f�r mechatronische Komponenten zu entwickeln.
Die Eignung f�r diesen Einsatz wurde im Rahmen eines Workshops f�r
Studenten und Wissenschaftler der TU M�nchen unter Beweiss
gestellt.  Im Rahmen dieses Workshops war es den Teilnehmern m�glich,
nach drei Tagen eigene mechatronische Komponenten zu entwickeln.

Das Projekt hat gezeigt, dass der Einsatz von Consumer Elektronik in
Bereichen, in denen normalerweise Spezialentwicklungen zum Einsatz
kommen, m�glich und auch sinnvoll ist.  Erm�glicht wurde dies nicht
zuletzt durch die Verbreitung von Linux auf nahezu allen relevanten
Rechnerarchitekturen.


% \section*{Links zu dem Artikel}
% \begin{description}
% \item[Linuxdevices] http://www.linuxdevices.com
% \item[Projekt L�ufer] http://www.projekt-laeufer.de
% \item[Cetron] http://www.cetron.de
% \item[Handhelds.org] http://www.handhelds.org
% \item[OPIE] http://opie.handhelds.org
% \end{description}