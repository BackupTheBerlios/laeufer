\documentclass[12pt, a4paper, twocolumn]{article}
\usepackage{german}           % Deutsches Layout
\usepackage[latin1]{inputenc} % Deutsche Sonderzeichen
%\usepackage{graphics}         % Grafiken an sich
%\usepackage{epsfig}           % EPS Bilder einbinden
\usepackage{hyperref}         % Links in PDF
\usepackage{palatino}         % Postscript Font
%\bibliographystyle{alphadin}
\bibliographystyle{alphadin}
%\bibliographystyle{abbrv}    % Alphadin stellt keine URLs dar
\author{Christoph Reichenbach, Christian R�diger, Markus Weimer}
\title{Strampelnder Tux}

\pagestyle{headings}

\begin{document}
\maketitle
\begin{abstract}
  \emph{Embedded Systems auf Linux-Basis sind aktuell eines der
    Boom-Gebiete des Open Source Systems.  Linux findet sich z.B. in
    der Steuerung von Industrieanlagen, auf kleinst-Webservern oder
    auch in j�ngster Zeit auf PDAs.  Eben diese Ger�te, die
    Pers�nlichen Digitalen Assistenten, finden seit einiger Zeit eine
    immer gr�ssere Verbreitung.  Im Rahmen des Projekts L�ufer wurden
    an der TU Darmstadt diese beiden Bereiche kleiner, eingebetteter
    Systeme zusammengef�hrt: Die Steuerung eines innovativen Fahrzeugs
    erfolgt durch einen Linux-PDA.}

\end{abstract}

% Inhaltsverzeichnis
%\tableofcontents

%Einleitung(mw)
%\section{Das Projekt L�ufer}

Bei diesem Fahrzeug handelt es sich um ein von seinen Entwicklern
,,L�ufer'' getauftes Tandem, das sich durch die Vielzahl an
technischen Neuerungen auszeichnet, deren Entwicklung sich zwei Zielen
unterordnet: Reisetauglichkeit und Fahrspa�.

% Projekt
Entwickelt wird das Fahrzeug von Studenten der TU Darmstadt, TU
M�nchen sowie der FH Darmstadt in einem sehr interdisziplin�ren Team.
Beteiligt sind neben Maschinenbauern verschiedener Spezialisierungen
auch (Wirtschafts-)Informatiker, Textil- und Industrie-Designer.  Ziel
des Projektes ist die Entwicklung eines muskelgetriebenen
\emph{Reise}fahrzeugs f�r zwei Personen.  Der Schwerpunkt liegt hier
auf der Reisetauglichkeit, und nicht auf Dingen wie der theoretisch
erreichbaren Maximalgeschwindigkeit.  Neben der technischen
Herausforderung sind das Lernen in anderen Strukturen sowie die
intensive Zusammenarbeit mit Partnern aus Industrie und Forschung
Triebfedern f�r das Projekt.

% Technik 
Ein wichtiger Schritt, die Ziele Fahrspa� und Reisetauglichkeit zu
erreichen, ist die Gewichtsreduktion. So ist der L�ufer komplett aus
Aluminium und Kohlefaser gefertigt, was sein Gesamtgewicht auf ca.
$50kg$ reduziert.  Der L�ufer verf�gt im Gegensatz zu vielen anderen
muskelgetriebenen Fahrzeugen �ber eine Verkleidung. Diese dient zum
einen der Verbesserung der Aerodynamik, zum anderen aber auch als
Regenschutz.  Dies zusammen mit dem Antriebsstrang, auf den weiter
unten eingegangen werden soll, erm�glicht es zwei normal trainierten
Menschen mit dem Fahrzeug eine Durchschnittsgeschwindigkeit von ca.
$40km/h$ zu erreichen.  Dieser errechnete Wert bezieht sich allerdings
nicht etwa auf eine besonders ebene Route, sondern auf eine Strecke im
Odenwald rund um Darmstadt, also im Mittelgebirge.


% �berleitung zu Christian
Insbesondere im Bereich der Mechatronik, also der Verbindung von
Mechanik mit moderner Informationstechnologie, geht der L�ufer nicht
nur f�r Fahrrad-Verh�ltnisse neue Wege.



%%% Local Variables:
%%% mode: latex 
%%% TeX-master: "master"
%%% End:

%Christian
% Auskommentiert, weils in nem Artikel bl�d kommt
%\section*{Mechatronik im L�ufer}
\subsection*{Mechatronische Komponenten im L�ufer}

Mechatronische Systeme am L�ufer sind der leistungsverzweigte
Hybridantrieb, das Energiespeichersystem, die intelligente Beleuchtung
sowie weitere funktionsrelevante Systeme.  Aber das System ist auch
f�r weitere Komponenten offen, die dann sehr einfach in die bestehende
Infrastruktur integriert werden k�nnen.  Im Projekt diskutiert wurden
die Erweiterung um ein GPS oder ein eigenes Display f�r den Beifahrer.

%MW: Der Folgende Satz ist zu lang. Ausserdem ``sollte es sein''
% Ziel der Mechatronik im L�ufer sollte es sein, beliebige Komponenten
% in das System integrieren zu k�nnen, ohne gr��ere �nderungen, vor
% allem baulicher Art, vornehmen zu m�ssen, sowie eine Schnittstelle f�r
% den Fahrer zur Verf�gung zu stellen.  Konkret galt es, eine ausfahrbare
% St�tze und ein Lichtsystem mit Blinkern, Warn- und Abblendlicht zu
% integrieren.  Aber auch weitere Komponenten, wie ein GPS oder ein
% zweites Display f�r den hinteren Fahrer sind denkbar.
%MW: Der Folgende Satz ist zu lang. Ausserdem ``sollte es sein''

F�r die Elektronik dieser mechatronischen Komponenten wurde in enger
Kooperation mit der Firma ce.tron\cite{cetron} eine Platine mit
vielf�ltigen Anschlussm�glichkeiten entwickelt. Zu diesen
M�glichkeiten geh�ren beispielsweise Anschl�sse f�r Motoren und
Sensorik. Die Steuerung der Anschl�sse erfolgt durch einen
PIC-kompatiblen Microcontroller.  Die mittels dieser Platine
gesteuerten mechatronischen Ger�te kommunizieren dann �ber einen
CANBus\cite{canspec}\footnote{Controller Area Network} mit der
zentralen Steuereinheit, also im Falle des L�ufers dem Linux-PDA.

\subsection*{Scheckkarten mit Funktion}

%MW: zu langer Satz
%Auf der Gr�sse einer Scheckkarte sind auf der Platine ein CANBus, ein
%Anschlu� f�r die serielle Schnittstelle, ein 12 Bit
%Analog-Digital-Wandler, zwei 12 Bit Digital-Analog-Wandler, ein 8-pin Digitaleingang, zwei
%8-Pin\footnote{pin: engl. Kontakt} Digitalein- oder Ausg�nge sowie
%zwei bis zu 36Volt belastbare Ausg�nge f�r Elektromotoren oder Servos
%untergebracht.  
%MW: zu langer Satz
%Der Microcontroller bekommt alle Werte �ber f�nf 8-Bit
%Ports\footnote{port: engl. Anschlu�} zur Verf�gung gestellt und kann
%�ber selbige auch die Ausg�nge steuern.

Soll nun eine neue Komponente in das System integriert werden, so
enth�lt die Platine bereits viele fertig implementierte Funktionen um
Eing�nge abzugreifen, auszuwerten und an den PDA weiterzuschicken.
Das Rad mu� also nicht jedes mal neu erfunden werden.  Mindestens die
Kommunikation mit anderen wird mit bereits auf der Platine
implementierten Funktionen abgewickelt. Der Entwickler mu� dazu nur
die von ihm ben�tigten unter den vorhandenen Routinen aufrufen.

Auf dieser untersten Ebene der Mechatronik sollen jedoch nur die
n�tigsten Entscheidungen getroffen werden.  Das Zusammenspiel der
Komponenten wir �ber den PDA kontolliert und ausgewertet.
%MW Hier ist ein reales Beispiel angebracht
% So w�rde
% z.B. ein Scheibenwischer zwar kontrollieren, ob er als n�chstes nach
% links oder recht wischen muss.  Ob er jedoch �berhaupt wischen soll,
% wird ihm �ber den CANBus vom PDA gesagt.
%MW Hier ist ein reales Beispiel angebracht
Ein Beispiel ist die Platine mit deren Hilfe die Fahrzeugst�tze des
L�ufers implementiert ist.  Sie schaltet den Motor zum St�tze
Ausfahren ab, wenn die Sensorik durch Beobachten von Stromspitzen am
Motor einen Widerstand registriert.  Ob die St�tze jedoch �berhaupt
ausgefahren werden sollwird im PDA entschieden und als Befehl �ber den
CANbus an diese Platine gesendet.

%%% Local Variables:
%%% mode: latex 
%%% TeX-master: "master"
%%% End:


%Markus
%\section*{PDA}

%�berleitung von Christian
All diese Komponenten des L�ufers m�chten die Fahrer nun zentral und
vor allem intuitiv steuern, um z.B. die Geschwindigkeit abh�ngig vom
eigenen Ersch�pfungsgrad und dem der Akkus einzustellen.  Aber nicht
nur dies, es m�ssen auch Fahrzeugdaten aufbereitet werden.  Diese
Informationen werden aus den Fahrzeugkomponenten via CANBus gewonnen,
und f�r den Fahrer in der Situation angemessen pr�sentiert.  Ein
einfaches Beispiel f�r solche Fahrdaten ist die aktuelle
Geschwindigkeit, die aus dem Antriebsstrang gewonnen wird.  Die
Anforderungen an das Fahrerinformationssystem des L�ufers sind also:


\begin{description}
\item[Kommunikation mit dem CANBus] Um die Informationen von den
  verschiedenen Ger�ten zu erhalten und diesen Befehle zu schicken.
\item[Rechenleistung] Um die Daten zu verarbeiten und die Berechnung
  komplexer Zusammenh�nge nicht in Assembler auf den Platinen
  durchf�hren zu m�ssen.
\item[Display] Um die Informationen an den Fahrer weiterzugeben.
  Diese Informationen sind z.B. bekannte Gr��en wie die aktuelle
  Geschwindigkeit, aber auch L�ufer-Spezifika wie der Ladestand der
  Batterien.
\end{description}

�ber Rechenleistung und Display verf�gen heutige PDAs in mehr als
ausreichendem Masse. Aus diesem Grund lag die Entscheidung nahe, einen
solchen als ,,Bordcomputer'' einzusetzen.  Nachdem erste Versuche mit
einem Palm IIIx sehr ermutigend waren, wurde das Team in der Ansicht
best�tigt, diesen Weg weiter zu verfolgen. Dies geschah entgegen der
verbreiteten Meinung, Consumer Elektronik habe im Embedded-Bereich
nichts verloren.

% ``Linux ist toll'' � Teil
Im Laufe der Entwicklung kam allerdings im Team der Wunsch nach einer
flexibleren L�sung als PalmOS auf.  Da PalmOS, das Betriebssystem der
Palm-PDAs, in seinen aktuellen Versionen nicht �ber Multitasking
verf�gt, h�tte jede Form von Erweiterung des L�ufers um neue
Komponenten �nderungen am Fahrprogramm erfordert.  Schlie�lich kann
man ohne Multitasking nicht einfach eine weitere Software
installieren, die ebenfalls w�hrend der Fahrt zur Ausf�hrung kommen
soll.  Eine Ab�nderung der sowieso schon laufenden Fahrsoftware sollte
aber aus Sicherheitsgr�nden vermieden werden.  Da das Projekt aber
seiner Struktur nach offen f�r neue Entwicklungen bleiben will, kam
ein ausschlie�en zuk�nftig hinzukommender Funktionalit�t nicht in
Frage.  So kam die Forderung nach einem Multitaskingf�higen PDA auf.
Zur Wahl standen die �blichen Verd�chtigen: Also WindowsCE/PocketPC
und seit neuestem auch Linux.

Das Mechatronik-Team in Darmstadt entschied sich f�r Linux, da es im
Bereich der Programmierung schon �ber mehr und tiefergehende
Erfahrungen mit Linux als mit WindowsCE verf�gte.  Ein weiterer
Vorteil von Linux auf dem PDA ist, da� die darauf laufenden
Anwendungen in der Regel Sourcecode-kompatibel zu Linux-PCs bleiben.
Man kann also eine Anwendung am PC entwickeln und testen, um sie dann
sp�ter durch einfaches Neu�bersetzen in eine PDA-Anwendung zu
verwandeln.  Dies ist ein nicht zu untersch�tzender Vorteil, gerade
vor dem Hintergrund der verteilten Produktentwicklung im Projekt.  Bei
dieser werden Teile der Mechatronik an der TU M�nchen entwickelt, wo
bei dieser Auslegung der Mechatronik kein PDA f�r die Entwicklung
n�tig ist.

Diese Methode kommt insbesondere auch ohne Emulation des PDA auf dem
PC aus.  Dies stellt sicher, da� die Anwendung auf dem PC eine genauso
direkte Verbindung zu den Schnittstellen hat wie sp�ter auf dem PDA,
ohne da� dem eine eventuelle Emulation des PDA eine weitere H�rde in
den Weg stellt.

Im Projekt wurde zun�chst ein mittels der Linux-Distribution Familiar
und der graphischen Oberfl�che OPIE \cite{opie} auf Linux umgestellter
Compaq IPaq H3660 verwendet.  \emph{SHARP!?!}

Programmiert werden beide PDAs in C++ unter Nutzung der
Klassenbibliothek QT des norwegischen Herstellers Trolltech.  Diese
Bibliothek liegt z.B. auch dem erfolgreichen UNIX-Desktop KDE zu
Grunde und erm�glicht es, Software f�r Windows, *ix, MacOS und nun
auch Linux-PDAs zu entwickeln.  Die dabei enstehende Software ist bei
gutem Programmierstil sogar durch einfaches Neu�bersetzen von einer
auf die andere Plattform portabel.  So wurde z.B. im Projekt L�ufer
Software f�r den PDA auf PCs und Workstations entwickelt, bevor der
erste PDA zur Verf�gung stand.  Bei der Gestaltung der Oberfl�che des
Programms mu� man dann allerdings die Einschr�nkungen des PDAs
hinsichtlich Texteingabe und Display-Aufl�sung ber�cksichtigen.

%�berleitung Jameson
%Im L�ufer mu�te dieser PDA an den sogenannten CANBus angebunden werden.


%%% Local Variables:
%%% mode: latex 
%%% TeX-master: "master"
%%% End:


% Jamesons Kaptitel
%\section{Die Verbindung zwischen PDA und Mechatronik}

All dies sind interessante Experimente f�r sich-- die Platine und ihre
Programmierung wie der PDA als Bedienelement-- doch damit das bunte
Bild der Bedienelemente nicht blosse Fa�ade bleibt und die Mechatronik
vom Willen des Fahrers erfahren kann, ist eine Kommunikationsschicht
n�tig.

%\subsection{Kommunikation}

F�r das Projekt L�ufer war von vorneherein die Entscheidung f�r ein
verkabeltes System gefallen.  Auf Protokollseite wird hierbei auf CAN
(Controller Area Network, siehe \cite{canspec}) gesetzt, ein
verh�ltnism��ig einfaches und verbreitetes Kommunikationsprotokoll,
das unter anderem von BMW und Mercedes zur Daten�bertragung innerhalb
von Fahrzeugen eingesetzt wird.

% IMG: CAN-Protokollframe
% Subtitel: Ein CAN-Frame, die atomare Kommunikationseinheit auf dem Bus

Ein unmittelbarer Vorteil der Wahl eines bereits verbreiteten
Protokolls ist das Vorhandensein existierender
Hardware-Implementierungen, von denen eine [-- FIXME: der sowieso-Chip
--] hier die Aufgabe eines Kommunikationsteilnehmers wahrimmt.


Das CAN-Protokoll gew�hrleistet Korrektheit der �bertragenen Daten,
erzwingt Arbitrierung der auf dem Bus gesandten Kommunikation und
erlaubt theoretisch 2048 logische und eine beliebige Anzahl
physikalischer Kommunikationsteilnehmer.

%\subsubsection{Erg�nzungsprotokolle}

Zwei weitere Probleme jedoch l�st CAN nicht: �bertragungssicherheit,
also die Gew�hrleistung, da� der Sender dem Erfolgen einer �bertragung
erf�hrt, ob diese beim Empf�nger ankam, und die �berpr�fung des
Vorhandenseins des vorgesehenen Bus-Masters, ein insbesondere durch
die problemlose Entfernbarkeit des PDAs relevantes Problem.


Die L�sungen erfolgten auf h�herer Protokollebene, die in Form des
eigens entworfenen LLO-Protokolles (L�ufer Layer One) entstand. Zwar
existiert mit CAN/Open (siehe \cite{canopen}) bereits ein etabliertes
und umfangreiches Standardprotokoll zur Kommunikation auf dem CAN-Bus,
das jedoch f�r eine Implementierung mit der gew�hlten Hardware und in
dem gegebenen Rahmen zu umfangreich gewesen w�re.

\subsection*{Die LLO- und LLZ-Protokolle}

LLO setzt direkt auf dem CAN-Protokoll auf und kodiert seine
Operationen in Teilen des CAN-Adressfeldes, so da� sich die Anzahl der
adressierbaren Ger�te auf 32 vermindert. Zwei dieser Adressen sind
wiederum reserviert, eine f�r den Bus-Master, eine f�r Broadcasts. Die
Notwendigkeiten des L�ufers fordern zur Zeit keinen gr��eren
Adressraum\footnote{Durch geeignete Versionnierung wurde jedoch
  sp�tere Erweiterbarkeit sichergestellt}.  Mittels eines regelm��igen
Keep-Alive-Signals stellt LLO sicher, da� die Sklaven im Bus einen
Ausfall des Bus-Masters bemerken; die andere noch fehlende
Anforderung, die Korrektheit von �bertragungen, wird allerdings durch
ein zweites Protokoll erf�llt: LLZ (L�ufer Layer Zero), konrastierend
zu den �blichen Konventionen benannt, liefert dies und auch eine
allgemeine Zustandsverwaltung der Busteilnehmer.

\subsubsection*{Ausgew�hlte Eingeweide}
%---- Ausgewaehlte Protokoll-Details

% Zustaende, sicherer Basiszustand
% Keep-Alive
Dabei werden vom Protokoll einige Zust�nde gefordert, insbesondere ein
stabiler Einzelbetriebszustand, in dem das angesteuerte Ger�t
eingehende Nachrichten ignoriert. Dieser wird eingenommen, wenn, zum
Beispiel durch mehrfaches Ausbleiben eines Keep-Alive-Signals vom
Busmaster, das Ger�t davon ausgehen mu�, da� es nicht mehr unter der
Aufsicht des PDAs steht und eine Fehlfunktion vorliegt. Wie genau
dieser sichere Betriebszustand tats�chlich aussieht, h�ngt vom
Ger�tetyp ab; der Motor beispielsweise nimmt in diesem Fall die
Drehzahl langsam zur�ck, eine vorhandene Warnblink-Anlage w�rde sich
automatisch aktivieren.

% Uebertragungssicherheit (Sequenzzaehler)
Zur Sicherstellung der korrekten �bertragung einer Botschaft werden
(auf LLO-Ebene) vierbittige Sequenzz�hler verwendet, die sich nach
jeder erfolgreichen Botschaft um eins inkreminieren. Da die Inhalte
der Z�hler in LLO-Botschaften einkodiert werden, kann der Empf�nger
bei Erhalt einer Nachricht seine Synchronit�t zum Bus-Master mit einem
hohen Grad an Sicherheit �berpr�fen.

\subsection*{Aus dem Tagebuch einer Nachricht}
%---- Weg einer Nachricht, mit Diagramm(en)

% Die Nachricht und ihre Semantik
%\begin{code}
%\end{code}


% Verpackung Highlevel-Protokoll
% Verpackung Lowlevel-Protokoll
% LL: An der Bushaltestelle
% LL: Der erste faehrt vorbei: Keepalive
% LL: Alles einsteigen, die Reise geht los
% LL: Gestrandet, aber keine Nausica�: Paket verloren
% LL: Unser Nachfolger kommt
% LL: Vermisstenmeldung: Der war doch direkt vor euch?
% LL: Suchtrupp, oder eher: Neuversandt eines Klons
% LL: Registrierungsnummer stimmt-- Welcome at the US Army Headquarters! Would you like to play a game?
% LL: Quittung fuer uns
% LL: Jacke ausziehen
% HL: Hose runter- der volle Monty!
% Wir haben unsre Semantik, und keiner kann zusehen, weil der Chip so klein und schwarz ist! Wie ein Zensierkaestchen... Hmm...


% Markus
\section*{Fazit}
Im Rahmen des Projekts L�ufer ist ein Mechatronik-Framework
entstanden, dass explizit von Informatikern f�r Ingenieure entwickelt
wurde.  Dies erm�glicht es den Ingenieuren im Projekt, ohne grossen
Aufwand die Software f�r mechatronische Komponenten zu entwickeln.
Die Eignung f�r diesen Einsatz wurde im Rahmen eines Workshops f�r
Studenten und Wissenschaftler der TU M�nchen unter Beweiss
gestellt.  Im Rahmen dieses Workshops war es den Teilnehmern m�glich,
nach drei Tagen eigene mechatronische Komponenten zu entwickeln.

Das Projekt hat gezeigt, dass der Einsatz von Consumer Elektronik in
Bereichen, in denen normalerweise Spezialentwicklungen zum Einsatz
kommen, m�glich und auch sinnvoll ist.  Erm�glicht wurde dies nicht
zuletzt durch die Verbreitung von Linux auf nahezu allen relevanten
Rechnerarchitekturen.


% \section*{Links zu dem Artikel}
% \begin{description}
% \item[Linuxdevices] http://www.linuxdevices.com
% \item[Projekt L�ufer] http://www.projekt-laeufer.de
% \item[Cetron] http://www.cetron.de
% \item[Handhelds.org] http://www.handhelds.org
% \item[OPIE] http://opie.handhelds.org
% \end{description}

% Zugefuegt --J
\bibliography{master}

\end{document}
